\documentclass[11pt, letterpaper]{article}
\usepackage[utf8]{inputenc}
\usepackage{hyperref}
\usepackage{enumerate}
\usepackage{listings}
\usepackage{fancyvrb}
\usepackage{kotex}
\usepackage{polski}
\usepackage[document]{ragged2e} % keep all left
\usepackage[english]{babel}

\usepackage{minted} % yaml syntax highlighting

\newenvironment{markdown}%
    {\VerbatimEnvironment\begin{VerbatimOut}{tmp.markdown}}%
    {\end{VerbatimOut}%
        \immediate\write18{pandoc tmp.markdown -t latex -o tmp.tex}%
        \input{tmp.tex}}

\providecommand{\tightlist}{%
  \setlength{\itemsep}{0pt}\setlength{\parskip}{0pt}}

\newcommand*{\email}[1]{\href{mailto:#1}{\nolinkurl{#1}} } 

\title{Kubernetes and Golang\\CloudNative Week\\{ \small \href{https://creativecommons.org/licenses/by/4.0/}{CC BY 4.0} }  }
\author{Wojciech Barczynski\\(wbarczynski.pro@gmail.com)}
\date{}


\begin{document}
\selectlanguage{english}

\begin{titlepage}
\maketitle
\end{titlepage}

\tableofcontents
\pagebreak
\section{Prerequiments}

\subsection{Audience}

We design the workshop with the following assumptions about the audience:

\begin{itemize}%
\item Can read and write Golang code, and understand its basics concepts.%
\item Have been working at lease 1 month with Kubernetes%
\item Feel good with Command Line Interface.%
\end{itemize}%

\subsection{Your workstation}

\begin{itemize}%
\item Linux or {\small OSX} recommended.%
\item Basic: \begin{itemize}%
    \item Golang,%
    \item a configured IDE or editor.%
    \item Git.
    \end{itemize}%
\item Kubernetes.%
\begin{itemize}%
    \item minikube,%
    \item kubectl.%
\end{itemize}
\item Docker: \begin{itemize}%
    \item Installed.%
    \item Account on hub.docker.com or quay.io.%
    \end{itemize}
\end{itemize}

%%%%%%%%%%%%%%%%%%%%%%%%%%%%%%%%%%%%%%%%%%%%%%%%%%%%%%%%%%%%%%%%%
\section{Check our dev environment}

Let's start a connection:

\begin{minted}{bash}
$ kubectl config use-context minikube
$ kubectl get po --all-namespaces
\end{minted}

You should see:

\begin{minted}{text}
NAMESPACE     NAME                     READY   STATUS    RESTARTS AGE
kube-system   coredns-5c98db65d4-kmz2d 1/1     Running   2        3d20h
kube-system   coredns-5c98db65d4-kwq6g 1/1     Running   2        3d20h
\end{minted}

%%%%%%%%%%%%%%%%%%%%%%%%%%%%%%%%%%%%%%%%%%%%%%%%%%%%%%%%%%%%%%%%%
\section{client-go for kubernetes}

1. Create your project

\begin{minted}{bash}
$ mkdir my-kube
$ cd my-kube
$ go mod init my-kube
\end{minted}

2. Le's get the library (\href{https://github.com/kubernetes/client-go}{https://github.com/kubernetes/client-go}):

\begin{minted}{bash}
$ go get k8s.io/client-go@kubernetes-1.15.3
\end{minted}

3. Write a program (based on \href{https://github.com/kubernetes/client-go/blob/master/examples/out-of-cluster-client-configuration/main.go}{example.go}):

%[frame=single]
\inputminted{go}{src/example_1.go}

Please change the program to display \mintinline{go}{Endpoints} and, later, to get all \mintinline{go}{Deployments}.

\bigskip
4. Please create in a golang program a service and ingress based on \href{https://github.com/kubernetes/client-go/blob/master/examples/create-update-delete-deployment/main.go}{this example}.

\bigskip
5. Let's see how to use \href{https://github.com/kubernetes/client-go/blob/master/testing/fake.go}{fake-client} for testing your program. Follow the instructor.

% testing: https://github.com/kubernetes/client-go/blob/master/examples/fake-client/main_test.go

% package main

% import (
% 	"testing"

% 	"k8s.io/client-go/kubernetes/fake"
% )

% // app
% type k8s struct {
%     clientset kubernetes.Interface
% }

% func (o *k8s) getVersion() (string, error) {
%     version, err := o.clientset.Discovery().ServerVersion()
%     if err != nil {
%         return "", err
%     }
%     return fmt.Sprintf("%s", version), nil
% }

% // -

% func newTestSimpleK8s() *k8s {
% 	client := k8s{}
% 	client.clientset = fake.NewSimpleClientset()
% 	return &client
% }

% func TestGetVersionDefault(t *testing.T) {
% 	k8s := newTestSimpleK8s()
% 	v, err := k8s.getVersion()
% 	if err != nil {
% 		t.Fatal("getVersion should not raise an error")
% 	}
% 	expected := "v0.0.0-master+$Format:%h$"
% 	if v != expected {
% 		t.Fatal("getVersion should return " + expected)
% 	}
% } https://itnext.io/testing-kubernetes-go-applications-f1f87502b6ef

%
% nice examples: https://medium.com/programming-kubernetes/building-stuff-with-the-kubernetes-api-part-4-using-go-b1d0e3c1c899
%%%%%%%%%%%%%%%%%%%%%%%%%%%%%%%%%%%%%%%%%%%%%%%%%%%%%%%%%%%%%%%%%
\section{operator-{\small SDK}}

1. Install operator {\small SDK} following instructions from \href{https://github.com/operator-framework/operator-sdk}{https://github.com/operator-framework/operator-sdk}. You can also download the binary from \href{https://github.com/operator-framework/operator-sdk/releases}{https://github.com/operator-framework/operator-sdk/releases}. 
\bigskip

2. Add the operator-sdk to your {\small PATH}.

\bigskip
3. Create a sample program, following quick start: \href{https://github.com/operator-framework/operator-sdk}{https://github.com/operator-framework/operator-sdk}. You might need to create an account on hub.docker.com or quay.io.

Notice: you might need to make your docker image public.

\bigskip
4. Let's extend change the command to:

\begin{minted}{go}
Spec: corev1.PodSpec{
    Containers: []corev1.Container{
        {
            Name:    "busybox2",
            Image:   "busybox",
            Command: []string{"/bin/sh","-c", "echo 'hello world' && sleep 3600"},
        },
    },
}
\end{minted}

\bigskip
5. We would like now our operator to ensure the command is as specified. Otherwise it should recreate the BusyBox. Let's add \verb|appservice_types.go|:

\begin{minted}{go}
type AppServiceSpec struct {
	BoxCommand []string `json:"command,omitempty"`
}
\end{minted}

and regenerate:

\begin{minted}{bash}
$ operator-sdk generate k8
$ operator-sdk generate openapi
\end{minted}

Change {\small CRD}:

\begin{minted}{bash}
$ cat deploy/crds/app_v1alpha1_appservice_cr.yaml
\end{minted}


Let's spcify it:

\begin{minted}{bash}
spec:
  command: [ "/bin/sh","-c", "echo 'hello world x2' && sleep 1000"]
  # Add fields here
  size: 3
\end{minted}

6. Check the {\small CR}:

\begin{minted}{bash}
$ kubectl get appservice
\end{minted}

% memcache tutorial looks good: https://github.com/operator-framework/operator-sdk/blob/master/doc/user-guide.md

%%%%%%%%%%%%%%%%%%%%%%%%%%%%%%%%%%%%%%%%%%%%%%%%%%%%%%%%%%%%%%%%%
\section{kubebuilder}

Kubebuilder (\href{https://github.com/kubernetes-sigs/kubebuilder}{https://github.com/kubernetes-sigs/kubebuilder}) is a framework for building Kubernetes {\small API}s.

% \bigskip
% 1. Install following \href{https://book.kubebuilder.io/quick-start.html#installation}{kubebuilder quick start}

\section{Observability}

Follow the instructor.

\begin{markdown}
- Monitoring - prometheus
- Logging - framework overview
\end{markdown}

See: rest service instruction, \href{https://github.com/wojciech12/talk_observability_logging}{github.com/wojciech12/talk\_observability\_logging}, and \href{https://github.com/wojciech12/talk_monitoring_with_prometheus}{github.com/wojciech12/talk\_monitoring\_with\_prometheus}.

%%%%%%%%%%%%%%%%%%%%%%%%%%%%%%%%%%%%%%%%%%%%%%%%%%%%%%%%%%%%%%%%%
\section{Golang app on kubernetes}

Which signals to handle, how to communicate with kubernetes?

See \href{https://github.com/wojciech12/talk_zero_downtime_deployment_with_kubernetes/tree/master/demo}{github.com/wojciech12/talk\_zero\_downtime\_deployment\_with\_kubernetes}



%%%%%%%%%%%%%%%%%%%%%%%%%%%%%%%%%%%%%%%%%%%%%%%%%%%%%%%%%%%%%%%%%
\section{References}

\begin{markdown}
- https://github.com/golang/go/wiki/CodeReviewComments
- https://golang.com/doc/effective_go.html
- http://devs.cloudimmunity.com/gotchas-and-common-mistakes-in-go-golang
\end{markdown}

\end{document}